\begin{center}
\Large \textbf{ABSTRACT}
\end{center}

\vspace{1cm}

TinkerBlocks is an educational tool designed to teach programming concepts to children through physical programming blocks and a programmable robotic car. The system combines physical learning with digital execution, allowing students to arrange physical blocks representing programming statements on a grid board to control a car that performs multiple tasks.

The project consists of three main components: a Raspberry Pi-based control system with multiple capabilities, a robotic car with advanced sensor integration, and a set of physical programming blocks. The Raspberry Pi performs image processing using OpenCV and EasyOCR to recognize block arrangements and instructions, interprets the visual programming language using an interpreter pattern, and controls the car through ESP32 communication.

The robotic car features differential steering with four DC motors, ultrasonic and IR sensors, an MPU-6050 gyroscope for precise movement and rotating, and a servo-controlled pen mechanism for drawing tasks.

Key technical achievements include the development of a good computer vision way for block recognition, implementation of an extensible command interpreter supporting loops, conditionals, and variables, creation of a reliable communication protocol between system components, and integration of multiple sensors for autonomous navigation and interaction detection.

This project successfully shows the feasibility of physical programming interfaces for educational purposes, providing an enjoyable way for children to learn programming concepts without the complexity of traditional text-based coding environments. Testing results show accurate block recognition, precise car movement control, and successful execution of complex programming logic including loops and conditional statements.

This project contributes to the growing field of educational robotics and physical programming, offering a scalable solution that can be adapted for various educational curricula and age groups.

\textbf{Keywords:} Educational Technology, Tangible Programming, Computer Vision, Robotics, Arduino, Raspberry Pi, Programming Education