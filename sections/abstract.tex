\begin{center}
    \Large \textbf{ABSTRACT}
\end{center}

\vspace{1cm}

Learning programming can be challenging for children due to abstract concepts and complex syntax. TinkerBlocks addresses this challenge by introducing a tangible programming approach where children arrange physical blocks on a grid to control a robotic car, making programming concepts fun and engaging.

This project developed an integrated system combining computer vision, robotics, and educational design. Children place programming blocks representing commands like movement, loops, and conditionals on a physical grid. A camera captures the arrangement, computer vision algorithms recognize the blocks and their positions, and the system translates this into executable code that controls a robotic car.

The implementation consists of three main components: a Raspberry Pi system that handles image processing and command interpretation, a robotic car equipped with sensors and actuators for autonomous movement, and a mobile application for more engaging features. The car can execute complex programs including loops, conditional statements, and sensor-based decisions while providing visual feedback through drawing capabilities.

Evaluation demonstrates successful block recognition with high accuracy, precise car movement control, and effective execution of programming concepts. Children can create programs ranging from simple movement sequences to complex algorithms involving variables and conditional logic.

TinkerBlocks successfully bridges the gap between abstract programming concepts and tangible interaction, providing an intuitive platform for programming education.

\textbf{Keywords:} Educational Technology, Computer Vision, Robotics, Programming Education