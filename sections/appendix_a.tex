\chapter{API DOCUMENTATION AND COMMAND REFERENCE}

\section{Arduino Serial API}

The Arduino firmware exposes a comprehensive serial API for controlling the robotic car. All commands follow the format: \texttt{command:\{"param1":"value1","param2":value2\}}

\subsection{Movement Commands}

\subsubsection{Move Command}

\begin{lstlisting}[language=json,caption=Move Command Examples]
// Move forward 20cm at speed 100
move:{"speed":100,"distance":20}

// Move backward for 1 second at speed 150
move:{"speed":-150,"timeMs":1000}

// Move 30cm in 2 seconds (speed calculated automatically)
move:{"distance":30,"timeMs":2000}

// Move without obstacle checking
move:{"speed":100,"distance":20,"checkUltrasonic":false}
\end{lstlisting}

\textbf{Parameters:}
\begin{itemize}
    \item \texttt{speed} (int): Motor speed (-255 to 255, negative for backward)
    \item \texttt{distance} (float): Distance in centimeters
    \item \texttt{timeMs} (unsigned long): Time in milliseconds
    \item \texttt{checkUltrasonic} (bool): Enable obstacle detection (default: true)
    \item \texttt{enableYawCorrection} (bool): Use gyro correction (default: true)
\end{itemize}

\textbf{Success Response:}
\begin{lstlisting}[language=json]
{
  "success": true,
  "success_result": "{\"distance_traveled\":20.00,\"time_taken\":784,\"final_yaw\":0.32}"
}
\end{lstlisting}

\textbf{Failure Response:}
\begin{lstlisting}[language=json]
{
  "success": false,
  "failure_reason": "Obstacle detected at 12.45cm"
}
\end{lstlisting}

\subsubsection{Rotation Command}

\begin{lstlisting}[language=json,caption=Rotation Command Examples]
// Turn left 90 degrees
rotate:{"angle":90,"speed":100}

// Turn right 45 degrees
rotate:{"angle":-45,"speed":80}

// Rotate to absolute heading (North = 0 degrees)
rotate:{"angle":0,"speed":100,"absolute":true}
\end{lstlisting}

\textbf{Parameters:}
\begin{itemize}
    \item \texttt{angle} (float): Rotation angle in degrees (positive = left/CCW, negative = right/CW)
    \item \texttt{speed} (int): Rotation speed (1-255, default: 100)
    \item \texttt{absolute} (bool): Absolute vs relative rotation (default: false)
\end{itemize}

\subsection{Sensor Commands}

\subsubsection{Ultrasonic Sensor}

\begin{lstlisting}[language=json,caption=Sensor Command Examples]
// Get distance reading
sensor:{"action":"distance"}

// Check for obstacles within 15cm
sensor:{"action":"obstacle","threshold":15}
\end{lstlisting}

\subsubsection{Gyroscope Commands}

\begin{lstlisting}[language=json,caption=Gyroscope Command Examples]
// Calibrate gyroscope
gyro:{"action":"calibrate"}

// Get current sensor data
gyro:{"action":"data"}

// Get current yaw angle
gyro:{"action":"yaw"}

// Set reference orientation
gyro:{"action":"reference"}
\end{lstlisting}

\subsection{Pen Control Commands}

\begin{lstlisting}[language=json,caption=Pen Control Examples]
// Lift pen up
pen:{"action":"up"}

// Put pen down
pen:{"action":"down"}

// Set custom pen position
pen:{"action":"position","position":45}
\end{lstlisting}

\section{ESP32 HTTP API}

The ESP32 serves as a WiFi bridge, exposing HTTP endpoints that translate to Arduino serial commands.

\subsection{HTTP Endpoints}

\begin{lstlisting}[language=bash,caption=HTTP API Endpoints]
# Movement control
POST /api/move
Content-Type: application/json
{"speed": 100, "distance": 20}

# Rotation control
POST /api/rotate
Content-Type: application/json
{"angle": 90, "speed": 100}

# Pen control
POST /api/pen
Content-Type: application/json
{"action": "up"}

# Sensor readings
POST /api/sensor
Content-Type: application/json
{"action": "distance"}

# Gyroscope control
POST /api/gyro
Content-Type: application/json
{"action": "calibrate"}

# IR sensor readings
POST /api/ir
Content-Type: application/json
{"action": "black_obstacle"}
\end{lstlisting}

\section{WebSocket Communication Protocol}

The Raspberry Pi control system uses WebSocket communication for real-time control and monitoring.

\subsection{Client Commands}

\begin{lstlisting}[language=json,caption=WebSocket Client Commands]
// Run complete OCR to engine pipeline
{
  "command": "run",
  "params": {
    "workflow": "full"
  }
}

// Run OCR only
{
  "command": "run",
  "params": {
    "workflow": "ocr_grid"
  }
}

// Run OCR with automatic engine execution
{
  "command": "run",
  "params": {
    "workflow": "ocr_grid",
    "chain_engine": true
  }
}

// Run engine with custom grid
{
  "command": "run",
  "params": {
    "workflow": "engine",
    "grid": [
      ["MOVE", "5"],
      ["TURN", "RIGHT"],
      ["LOOP", "3"],
      ["", "MOVE", "2"],
      ["", "TURN", "LEFT"]
    ]
  }
}

// Stop current process
{
  "command": "stop"
}
\end{lstlisting}

\subsection{Server Responses}

\begin{lstlisting}[language=json,caption=WebSocket Server Responses]
// Status update
{
  "type": "status",
  "message": "Processing image..."
}

// Progress update
{
  "type": "progress",
  "percentage": 45,
  "message": "Running OCR..."
}

// Completion notification
{
  "type": "complete",
  "result": "Success",
  "data": {
    "execution_time": 2.3,
    "commands_executed": 15
  }
}

// Error notification
{
  "type": "error",
  "message": "Block recognition failed",
  "details": "Insufficient lighting detected"
}
\end{lstlisting}

\section{Programming Language Reference}

\subsection{Movement Commands}

\begin{lstlisting}[caption=Movement Command Examples]
// Basic movement
MOVE                    // Move forward 1 unit
MOVE | 5               // Move forward 5 units
MOVE | -3              // Move backward 3 units
MOVE | X               // Move forward X units (variable)

// Conditional movement
MOVE | WHILE | X < 10  // Move while X is less than 10
MOVE | WHILE | DISTANCE > 30  // Move while distance > 30cm

// Turn commands
TURN | LEFT                      // Turn left 90 degrees
TURN | RIGHT                     // Turn right 90 degrees
TURN | 45                        // Turn right 45 degrees
TURN | LEFT | 30                 // Turn left 30 degrees
TURN | RIGHT | WHILE | X < 5     // Turn right while X < 5
\end{lstlisting}

\subsection{Control Flow Examples}

\begin{lstlisting}[caption=Control Flow Examples]
// Simple loop
LOOP | 5
    MOVE | 2
    TURN | RIGHT

// Conditional loop
LOOP | WHILE | X < 10
    MOVE | 1
    SET | X | X + 1

// Infinite loop with condition
LOOP | TRUE
    MOVE | 1
    IF | OBSTACLE
        TURN | RIGHT

// Conditional statements
IF | DISTANCE < 30
    TURN | LEFT
ELSE
    MOVE | 5

// Nested conditions
IF | X > 10
    IF | Y < 5
        MOVE | X
    ELSE
        TURN | RIGHT
ELSE
    SET | X | 0
\end{lstlisting}

\subsection{Variable Operations}

\begin{lstlisting}[caption=Variable Operation Examples]
// Basic assignment
SET | X | 5
SET | SPEED | 10
SET | Y | DISTANCE
SET | FLAG | TRUE

// Arithmetic operations
SET | Z | X + 2
SET | COUNT | COUNT + 1
SET | RESULT | X * Y / 2
SET | AVERAGE | (X + Y) / 2

// Boolean operations
SET | FOUND | DISTANCE < 30
SET | SAFE | NOT OBSTACLE
SET | READY | X > 0 AND Y > 0
SET | CONTINUE | DISTANCE > 10 OR COUNT < 5
\end{lstlisting}

\subsection{Sensor Integration}

\begin{lstlisting}[caption=Sensor Integration Examples]
// Using sensor values in conditions
IF | DISTANCE < 20
    TURN | RIGHT

IF | OBSTACLE
    MOVE | -2
    TURN | 180

IF | BLACK_DETECTED
    MOVE | 1
ELSE
    TURN | LEFT

// Sensor values in expressions
SET | SAFE_DISTANCE | DISTANCE + 5
SET | TOO_CLOSE | DISTANCE < 15

// Waiting for sensor conditions
WAIT | WHILE | DISTANCE < 20
WAIT | WHILE | BLACK_DETECTED
\end{lstlisting}

\subsection{Drawing Commands}

\begin{lstlisting}[caption=Drawing Command Examples]
// Draw a square
SET | SIDE | 3
PEN_DOWN
LOOP | 4
    MOVE | SIDE
    TURN | RIGHT
PEN_UP

// Draw with conditional pen control
LOOP | 10
    IF | X % 2 = 0
        PEN_DOWN
    ELSE
        PEN_UP
    MOVE | 2
    SET | X | X + 1
\end{lstlisting}

\section{Error Codes and Troubleshooting}

\subsection{Common Error Codes}

\begin{lstlisting}[caption=Error Code Reference]
// Arduino Serial API Errors
"Invalid speed. Must be between 1 and 255."
"Obstacle detected at 12.45cm"
"Missing or invalid 'position' parameter for pen position"
"Unknown gyro action: invalid_action"
"Unknown sensor action: invalid_action"

// Vision Processing Errors
"Block recognition failed - insufficient lighting"
"Grid detection failed - corners not found"
"OCR processing timeout"
"Invalid grid dimensions"

// Engine Execution Errors
"Unknown command: INVALID_COMMAND"
"Maximum steps exceeded (1000)"
"Variable 'X' not defined"
"Division by zero in expression"
"ELSE without matching IF"
\end{lstlisting}

\subsection{Troubleshooting Guide}

\begin{lstlisting}[caption=Common Solutions]
// Poor block recognition
Problem: Low recognition accuracy
Solution: 
- Ensure lighting > 300 lux
- Check camera focus and positioning
- Verify block text clarity
- Clean camera lens

// Communication timeouts
Problem: API timeouts
Solution:
- Check WiFi connection strength
- Verify ESP32 power supply
- Restart system components
- Check serial cable connections

// Movement inaccuracy
Problem: Car movement imprecise
Solution:
- Calibrate gyroscope sensor
- Check battery voltage level
- Verify wheel alignment
- Test on smooth surface

// Program execution errors
Problem: Commands not executing
Solution:
- Verify grid syntax correctness
- Check indentation alignment
- Validate parameter ranges
- Review variable definitions
\end{lstlisting}