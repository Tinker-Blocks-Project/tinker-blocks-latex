\chapter{INTRODUCTION}

\section{Background and Motivation}

In today's digital generation, programming literacy has become as fundamental as traditional literacy \cite{wing2006computational}. However, teaching programming concepts to children, specifcally those in elementary and middle school, shows significant challenges. Traditional text-based programming environments can be hard for young learners, making barriers to understanding fundamental computational thinking concepts.

The concept of tangible programming interfaces has showed as a promising solution to fill this gap \cite{ishii2008tangible}. By providing physical objects that represent programming instructions, these systems make abstract concepts concrete. This approach aligns with learning theories, which emphasize learning through hands-on exploration and manipulation of physical objects.

Educational robotics has proven to be an effective tool for engaging students in learning \cite{benitti2012exploring}. The immediate visual and physical feedback provided by robotic systems helps students understand the consequences of their programming decisions. However, most existing educational robotics platforms still rely on screen-based programming interfaces, which may not be suitable for younger learners or those who learn better through physical interaction.

\section{Problem Statement}

Current approaches to programming education for children face several limitations:

\begin{enumerate}
    \item \textbf{Abstract Interface}: Traditional programming environments use text-based interfaces that can be hard and not enjoyable for young learners.
    \item \textbf{Limited Tactile Interaction}: Most educational programming tools are screen-based, lacking the physical manipulation that enhances learning for  learners.
    \item \textbf{Complexity Barrier}: Existing systems often require prior knowledge of syntax and programming concepts.
    \item \textbf{Limited Engagement}: Text-based programming may not sustain the attention and interest of young learners.
    \item \textbf{Accessibility Issues}: Traditional programming interfaces may not be suitable for learners with certain learning differences or disabilities.
\end{enumerate}

\section{Proposed Solution}

TinkerBlocks addresses these challenges by providing a tangible programming interface that combines physical programming blocks with a robotic car. The system allows children to:

\begin{itemize}
    \item Arrange physical blocks on a grid to create programs
    \item See immediate results through car movement and drawing
    \item Learn programming concepts through hands-on manipulation
    \item Progress from simple movements to complex algorithms
    \item Engage in collaborative programming activities
\end{itemize}

The system uses computer vision to interpret block arrangements and translates them into executable commands for the robotic car, creating an easy bridge between physical manipulation and digital execution.

\section{Objectives}

The primary objectives of this project are:

\subsection{Primary Objectives}
\begin{enumerate}
    \item Design and implement a tangible programming interface using physical blocks
    \item Develop a computer vision system for accurate block recognition and grid mapping
    \item Create a programmable robotic car with advanced sensor integration
    \item Implement a robust interpreter for executing visual programming languages
\end{enumerate}


\section{Scope and Limitations}

\subsection{Scope}
This project encompasses:
\begin{itemize}
    \item Development of hardware components (robotic car, sensors, control board)
    \item Implementation of computer vision algorithms for block recognition
    \item Creation of a command interpreter for program execution
    \item Design of multiple interactive game modes
    \item Integration of all system components through communication protocols
    \item Testing and validation of system functionality
\end{itemize}

\subsection{Limitations}
The current implementation has the following limitations:
\begin{itemize}
    \item Limited to a 16x10 grid size for program complexity
    \item Requires controlled lighting conditions for optimal computer vision performance
    \item Single car operation (no multi-robot scenarios)
    \item Limited to predefined programming constructs
    \item Requires manual block placement and arrangement
\end{itemize}

\section{Report Structure}

This report is organized as follows:

\textbf{Chapter 2} presents a comprehensive literature review of related work in educational robotics, tangible programming interfaces.

\textbf{Chapter 3} details the system design and architecture, including hardware specifications, software architecture, and communication protocols.

\textbf{Chapter 4} describes the implementation of all system components, including the computer vision, command interpreter, and robotic car firmware.

\textbf{Chapter 5} concludes the report with a summary of achievements, contributions, and recommendations for future work.

The appendices provide detailed technical information including API documentation, code examples, and system specifications.