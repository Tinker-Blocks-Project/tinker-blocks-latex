\chapter{LITERATURE REVIEW}

\section{Educational Robotics}

Educational robotics has shown as a powerful tool for teaching STEM concepts, particularly programming and computational thinking \cite{benitti2012exploring}. The field has evolved significantly since the introduction of Logo and the turtle graphics system in the 1960s, which first demonstrated the potential of robotics in education.

Modern educational robotics platforms like LEGO Mindstorms, VEX Robotics, and Arduino-based systems have made robotics more accessible to students and educators.\cite{papert1980mindstorms} These platforms typically combine programmable microcontrollers with sensors, actuators, and mechanical components, allowing students to build and program their own robots.

Research has shown that educational robotics can improve student thinking, problem-solving skills, and understanding of programming concepts  \cite{sullivan2013robots}. The immediate visual and touchable feedback provided by robotic systems helps students understand abstract programming concepts and debug their code more effectively.

\section{Tangible Programming Interfaces}

Tangible programming interfaces represent a paradigm shift from traditional screen-based programming environments to physical manipulation of programming constructs \cite{ishii2008tangible}. This approach is rooted in the theory of embodied cognition, which suggests that physical interaction with objects enhances learning and understanding.

\subsection{Historical Development}
The concept of tangible programming can be traced back to early educational toys like the Big Trak programmable robot and more recent systems like the AlgoBlock system developed at MIT \cite{bau2009alphablocks}. These early systems demonstrated that children could understand programming concepts through physical manipulation long before they could master text-based programming languages \cite{suzuki1999interaction}.

\subsection{Current Approaches}

Several systems have explored tangible programming interfaces:

\begin{itemize}
    \item \textbf{Cubetto}: A wooden robot that uses colored blocks to represent programming commands
    \item \textbf{Code \& Go}: A board game approach to teaching programming logic
    \item \textbf{Osmo Coding}: Uses physical blocks detected by tablet cameras
    \item \textbf{KIBO}: A construction kit for children to build and program robots using wooden blocks \cite{kazakoff2011putcode}
\end{itemize}

\subsection{Advantages and Challenges}

Tangible programming interfaces offer several advantages:
\begin{itemize}
    \item Reduced cognitive load by eliminating syntax requirements
    \item Enhanced spatial understanding of program flow
    \item Support for collaborative programming activities
    \item Accessibility for learners with different abilities
\end{itemize}

However, they also present challenges:
\begin{itemize}
    \item Limited scalability for complex programs
    \item Physical constraints on program size
    \item Recognition accuracy in computer vision-based systems
    \item Higher hardware costs compared to software-only solutions
\end{itemize}

\section{Computer Vision in Educational Applications}

Computer vision has become increasingly important in educational technology, enabling systems to interpret and respond to physical student interactions. In the context of tangible programming interfaces, computer vision serves as the bridge between physical manipulation and digital execution.


\subsection{Constructivist Learning Theory}

Based on the work of Jean Piaget and Seymour Papert, constructivist learning theory emphasizes learning through construction and experimentation. This theory has been particularly influential in the design of programming tools for children, promoting hands-on exploration over direct instruction.

\subsection{Block-Based Programming Languages}
Visual programming languages like Scratch, Blockly, and Alice have revolutionized programming education by providing drag-and-drop interfaces that eliminate syntax errors and focus on logic and computational thinking \cite{resnick2009scratch}. These tools have demonstrated the effectiveness of visual approaches to programming education \cite{maloney2010scratch}

\subsection{Computational Thinking}

Computational thinking encompasses four key skills:
\begin{itemize}
    \item \textbf{Decomposition}: Breaking complex problems into smaller parts
    \item \textbf{Pattern Recognition}: Identifying similarities and patterns
    \item \textbf{Abstraction}: Focusing on essential features while ignoring irrelevant details
    \item \textbf{Algorithm Design}: Creating step-by-step solutions
\end{itemize}

Effective programming education tools should support the development of these skills through appropriate progression.

\section{Related Systems and Technologies}

\subsection{Commercial Educational Robotics Platforms}

Several commercial platforms share similarities with TinkerBlocks:

\textbf{LEGO Mindstorms:} A comprehensive robotics platform that combines programmable bricks with sensors and motors. While highly capable, it relies on screen-based programming interfaces and requires significant investment in components.

\textbf{Bee-Bot and Blue-Bot:} Simple programmable robots designed for early childhood education. These systems use button-based programming but lack the flexibility of a visual programming interface.

\textbf{Dash and Dot:} Smartphone-controlled robots with visual programming apps. While accessible, they depend on mobile devices and don't provide tangible programming experiences.

\subsection{Research Prototypes}

Academic research has produced several innovative approaches to tangible programming:

\textbf{Tern:} A tangible programming system that uses physical tiles arranged on a surface, detected by an overhead camera. This system demonstrated the feasibility of camera-based block recognition.

\textbf{Topobo:} A 3D construction kit with kinetic memory, allowing children to teach robots behaviors through physical demonstration.

\textbf{FlowBlocks:} A tangible programming system for controlling robotic devices using magnetic blocks that can be arranged on a whiteboard.

\section{Gaps in Current Research}

Despite significant progress in educational robotics and tangible programming interfaces, several gaps remain:

\begin{enumerate}
    \item \textbf{Limited Integration:} Most systems focus on either robotics or programming, but few successfully integrate both in a easy learning experience.
    
    \item \textbf{Scalability Issues:} Many tangible programming systems are limited in the complexity of programs they can represent.
    
    \item \textbf{Curriculum Integration:} Most systems exist as standalone tools rather than integrated components of educational curricula.
    
    \item \textbf{Accessibility:} Limited research has been conducted on making these systems accessible to learners with different abilities and learning styles.
\end{enumerate}

\section{Positioning of TinkerBlocks}

TinkerBlocks addresses several of these gaps by:

\begin{itemize}
    \item Providing seamless integration between tangible programming and robotic execution
    \item Supporting multiple game modes that aims to different learning objectives
    \item Using modern computer vision techniques for robust block recognition
    \item Implementing a scalable interpreter architecture that can support complex programming constructs
    \item Designing a modular system that can be extended and adapted for different educational contexts
\end{itemize}

The system builds upon the strengths of existing approaches while addressing their limitations through innovative technical solutions and design decisions.