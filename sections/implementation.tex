\chapter{IMPLEMENTATION}

\section{Development Methodology}

The TinkerBlocks system was developed using an iterative prototyping methodology, allowing for continuous refinement and validation throughout the development process. The implementation followed a modular approach, with each component developed and tested independently before integration.

\subsection{Development Phases}

\subsubsection{Phase 1: Core Infrastructure}
The initial phase focused on establishing the foundational architecture:
\begin{itemize}
    \item Implementation of the core module with WebSocket server
    \item Development of the process controller and workflow framework
    \item Creation of configuration management system
    \item Establishment of communication protocols
\end{itemize}

\subsubsection{Phase 2: Computer Vision Pipeline}
The second phase concentrated on the vision processing capabilities:
\begin{itemize}
    \item Integration of camera capture systems
    \item Development of grid detection algorithms
    \item Implementation of OCR processing pipeline
    \item Creation of block-to-grid mapping system
\end{itemize}

\subsubsection{Phase 3: Command Interpreter}
The third phase implemented the programming language interpreter:
\begin{itemize}
    \item Design of command registry and factory patterns
    \item Implementation of expression evaluation system
    \item Development of control flow constructs
    \item Creation of variable management system
\end{itemize}

\subsubsection{Phase 4: Hardware Integration}
The final phase integrated all hardware components:
\begin{itemize}
    \item Arduino firmware development
    \item ESP32 WiFi bridge implementation
    \item Sensor integration and calibration
    \item Motor control and movement algorithms
\end{itemize}


\subsubsection{Phase 5: Mobile Application Development}
The final phase implemented the user interface and mobile application:
\begin{itemize}
    \item React Native Expo application development
    \item WebSocket communication integration
    \item Modern UI design with smooth animations
    \item Cross-platform compatibility testing
\end{itemize}

\section{Core Module Implementation}

\subsection{WebSocket Server}

The WebSocket server provides real-time communication between system components using asyncio-based architecture. The implementation features:

\begin{itemize}
    \item Asynchronous connection handling for multiple clients
    \item JSON-based message protocol with command routing
    \item Broadcast functionality for system-wide notifications
    \item Error handling and connection recovery mechanisms
\end{itemize}

Key implementation features include automatic connection management, message queuing for reliability, and extensible command processor registration.

\subsection{Process Controller}

The process controller manages workflow execution with support for cancellation and progress monitoring:

\begin{itemize}
    \item Generic workflow protocol supporting any async function
    \item Cancellation support through callback mechanisms
    \item Progress reporting via message callbacks
    \item Return value handling for workflow chaining
\end{itemize}

The implementation uses Python's asyncio primitives to ensure responsive execution while maintaining system stability.

\subsection{Configuration Management}

Centralized configuration using Pydantic models provides type-safe settings management:

\begin{itemize}
    \item Immutable configuration objects with validation
    \item Environment variable support for deployment flexibility
    \item Default values for all settings
    \item Type checking and automatic conversion
\end{itemize}

\section{Vision Module Implementation}

\subsection{Image Processing Pipeline}

The vision module implements a comprehensive image processing pipeline for block recognition and grid mapping.

\subsubsection{Image Capture}

Multiple capture modes support different deployment scenarios:

\begin{itemize}
    \item \textbf{Local Capture}: Direct camera access via OpenCV for development
    \item \textbf{Remote Capture}: Client-server architecture for Raspberry Pi deployment
    \item \textbf{File-based Testing}: Static image processing for testing and validation
\end{itemize}

The capture system automatically handles image rotation, format conversion, and quality optimization.

\subsubsection{Grid Detection}

Perspective transformation enables accurate grid detection despite camera positioning variations:

\begin{itemize}
    \item Automatic corner detection using configurable reference points
    \item Perspective correction with homography transformation
    \item Grid cell boundary calculation for precise block placement
    \item Adaptive scaling for different grid sizes
\end{itemize}

The implementation uses OpenCV's perspective transformation functions with custom algorithms for grid cell mapping.

\subsubsection{OCR Processing}

EasyOCR integration provides robust text recognition with GPU acceleration:

\begin{itemize}
    \item Multi-language support with English optimization
    \item Confidence scoring for recognition quality assessment
    \item Bounding box detection for precise text localization
    \item GPU acceleration when available for improved performance
\end{itemize}

Alternative OCR engines can be integrated through the modular wrapper design.

\subsection{Block-to-Grid Mapping}

The OCR2Grid mapper translates detected text into structured grid data:

\begin{itemize}
    \item Spatial analysis to determine grid cell associations
    \item Confidence-based filtering to eliminate false positives
    \item Multi-word block support for complex commands
    \item Empty cell detection and handling
\end{itemize}

The mapping algorithm uses geometric analysis to associate OCR results with specific grid positions, handling overlapping text and partial occlusion.

\section{Engine Module Implementation}

\subsection{Command System Architecture}

The engine implements a comprehensive command system using the interpreter pattern:

\subsubsection{Command Registry}

A factory pattern enables dynamic command registration and creation:

\begin{itemize}
    \item Automatic command discovery through module imports
    \item Type-safe command creation with parameter validation
    \item Extensible command system for future additions
    \item Error handling for unknown or malformed commands
\end{itemize}

Each command class implements a common interface for parsing, validation, and execution.

\subsubsection{Parser Implementation}

The parser converts 2D grid data into an executable command tree:

\begin{itemize}
    \item Left-to-right, top-to-bottom grid traversal
    \item Indentation-based scope detection using column positions
    \item Special handling for ELSE statements in conditional blocks
    \item Argument collection and parameter binding
\end{itemize}

The parsing algorithm maintains a stack-based approach for handling nested structures while preserving execution order.

\subsection{Value System}

A comprehensive value system supports different data types and expressions:

\subsubsection{Base Value Types}

\begin{itemize}
    \item \textbf{Number}: Integer and floating-point numeric values
    \item \textbf{Boolean}: True/false logical values
    \item \textbf{Variable}: Named storage with dynamic typing
    \item \textbf{Direction}: Spatial orientation values (LEFT, RIGHT, etc.)
    \item \textbf{Sensor}: Real-time sensor data access
\end{itemize}

\subsubsection{Expression Evaluation}

Complex expressions support mathematical and logical operations:

\begin{itemize}
    \item Left-to-right evaluation with operator precedence
    \item Type coercion for mixed-type operations
    \item Short-circuit evaluation for logical operators
    \item Variable resolution with scope management
\end{itemize}

The expression evaluator handles nested operations while maintaining performance through efficient parsing and caching.

\subsection{Execution Context}

The execution context maintains program state throughout execution:

\begin{itemize}
    \item Position and orientation tracking for the car
    \item Variable storage with scope management
    \item Drawing state for pen-based operations
    \item Step counting for execution limits
    \item Path tracking for visualization
\end{itemize}

Context state is immutable where possible to support debugging and replay functionality.

\subsection{Control Flow Implementation}

\subsubsection{Loop Constructs}

Multiple loop types support different programming patterns:

\begin{itemize}
    \item \textbf{Count Loops}: Fixed iteration with numeric parameters
    \item \textbf{Conditional Loops}: While loops with dynamic condition evaluation
    \item \textbf{Infinite Loops}: Bounded by maximum step limits
    \item \textbf{Boolean Loops}: Direct boolean value evaluation
\end{itemize}

Loop implementation includes proper nesting support and break conditions to prevent infinite execution.

\subsubsection{Conditional Execution}

IF/ELSE statements provide branching logic:

\begin{itemize}
    \item Dynamic condition evaluation using the value system
    \item Proper ELSE handling at matching indentation levels
    \item Nested conditional support for complex logic
    \item Short-circuit evaluation for performance
\end{itemize}

\section{Hardware Implementation}

\subsection{Arduino Firmware}

The Arduino firmware provides low-level hardware control with a structured API:

\subsubsection{Class-Based Architecture}

\begin{itemize}
    \item \textbf{Motor Class}: Direct hardware interaction for individual motors
    \item \textbf{MotionController}: High-level movement coordination
    \item \textbf{GyroSensor}: IMU integration for precise orientation
    \item \textbf{UltrasonicSensor}: Distance measurement and obstacle detection
    \item \textbf{PenController}: Servo-based drawing mechanism control
\end{itemize}

Each class encapsulates specific functionality while providing clean interfaces for integration.

\subsubsection{Movement Algorithms}

Advanced movement control ensures precision and reliability:

\begin{itemize}
    \item \textbf{Yaw Correction}: Gyroscope-based straight-line movement
    \item \textbf{PID Control}: Precise rotation with feedback control
    \item \textbf{Obstacle Avoidance}: Real-time UltraSonic sensor integration
\end{itemize}

The movement algorithms balance precision with smooth operation, using sensor feedback for continuous correction.

\subsubsection{Sensor Integration}

Multiple sensors provide comprehensive environmental awareness:

\begin{itemize}
    \item \textbf{MPU-6050}: 6-axis IMU for orientation and acceleration
    \item \textbf{HC-SR04}: Ultrasonic ranging for obstacle detection
    \item \textbf{IR Sensors}: Black Circle detection for reaching it or not.
    \item \textbf{Servo Feedback}: Pen position monitoring
\end{itemize}

Sensor data is filtered and processed to provide reliable input for control algorithms.

\subsection{ESP32 WiFi Bridge}

The ESP32 serves as a communication bridge between the Raspberry Pi and Arduino:

\subsubsection{HTTP API Server}

A RESTful API provides remote control capabilities:

\begin{itemize}
    \item JSON-based request/response format
    \item Comprehensive error handling and status reporting
    \item Timeout management for reliable operation
    \item Command validation and parameter checking
\end{itemize}

The API design follows REST principles while optimizing for the specific use case of robotic control.

\subsubsection{Serial Communication}

Reliable serial communication with the Arduino:

\begin{itemize}
    \item 115200 baud UART communication
    \item JSON protocol for command and response exchange
    \item Error detection and retry mechanisms
    \item Flow control for data integrity
\end{itemize}

\subsection{Power Management}

Efficient power management ensures reliable operation:

\begin{itemize}
    \item Voltage regulation for stable 5V logic supply
    \item Battery monitoring and low-voltage protection
    \item Sleep modes for power conservation
    \item Graceful shutdown procedures
\end{itemize}
The implementation successfully delivers a robust, scalable, and maintainable system that meets all design objectives while providing room for future enhancements and extensions.